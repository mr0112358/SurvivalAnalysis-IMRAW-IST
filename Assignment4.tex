\documentclass[11pt]{article}

\usepackage{amsmath, amssymb, amsthm, bm, bbm,graphicx, mathtools, enumerate,multirow}
\usepackage[letterpaper, left=1.1truein, right=1.1truein, top = 1.1truein,
bottom = 1.1truein]{geometry}
\usepackage[affil-it]{authblk}
\usepackage{natbib}
\usepackage{hyperref}
\usepackage[usenames,dvipsnames]{color}
\usepackage[ruled, vlined, lined, commentsnumbered]{algorithm2e}
\usepackage{prettyref,soul}
\usepackage{float}
\usepackage{setspace}
\usepackage{listings}

\lstset{ 
  language=R,                     % the language of the code
  basicstyle=\tiny\ttfamily, % the size of the fonts that are used for the code
  numbers=left,                   % where to put the line-numbers
  numberstyle=\tiny\color{Blue},  % the style that is used for the line-numbers
  stepnumber=1,                   % the step between two line-numbers. If it is 1, each line
                                  % will be numbered
  numbersep=5pt,                  % how far the line-numbers are from the code
  backgroundcolor=\color{white},  % choose the background color. You must add \usepackage{color}
  showspaces=false,               % show spaces adding particular underscores
  showstringspaces=false,         % underline spaces within strings
  showtabs=false,                 % show tabs within strings adding particular underscores
  frame=single,                   % adds a frame around the code
  rulecolor=\color{black},        % if not set, the frame-color may be changed on line-breaks within not-black text (e.g. commens (green here))
  tabsize=2,                      % sets default tabsize to 2 spaces
  captionpos=b,                   % sets the caption-position to bottom
  breaklines=true,                % sets automatic line breaking
  breakatwhitespace=false,        % sets if automatic breaks should only happen at whitespace
  keywordstyle=\color{RoyalBlue},      % keyword style
  commentstyle=\color{YellowGreen},   % comment style
  stringstyle=\color{ForestGreen}      % string literal style
} 

\newcommand{\numit}{\stepcounter{equation}\tag{\theequation}}


\DeclareMathOperator*{\argmin}{arg\,min}
\newcommand{\E}{\mbox{{\rm E}}}
\newcommand{\Var}{\mbox{{\rm Var}}}
\newcommand{\tr}{\mbox{{\rm tr}}}
\newcommand{\diag}{\mbox{{\rm diag}}}

\newtheorem{thm}{Theorem}
\newtheorem{defi}{Definition}
\newtheorem{lem}{Lemma}
\newtheorem{coro}{Corollary}
\newtheorem{prop}{Proposition}
\newtheorem{ex}{Example}
\newtheorem{rmk}{Remark}
\newtheorem{asmp}{Assumption}

\newcommand{\hS}{\hat{S}}
\newcommand{\hsS}{\hat{S}^{*}}
\newcommand{\hlam}{\hat{\lambda}}
\newcommand{\hLam}{\hat{\Lambda}}
\newcommand{\htheta}{\hat{\theta}}

\newcommand{\cS}[1]{S \left( #1 \mid \bZ_i \right)}
\newcommand{\cf}[1]{f \left( #1 \mid \bZ_i \right)}
\newcommand{\clamb}[1]{\lambda \left( #1 \mid \bZ_i \right)}

\newcommand{\bT}{{\bf{T}}}
\newcommand{\bX}{{\bf{X}}}
\newcommand{\bZ}{{\bf{Z}}}
\newcommand{\bdelta}{{\bm \delta}}
\newcommand{\bbeta}{{\bm \beta}}



\begin{document}

\title{STAT 6227, Assignment \#4}
\maketitle

\section{Description}
We consider the IMRAW-IST dataset studied in Sloand et al., (Jouranal of
Clinical Oncology, 2008, Vol. 26, No. 15, 2505-2511) and would like to model the
hazard rate for time-to-death $T$ based on the observed censored data
$\{(T_i,\delta_i):i=1,\dots,n\}$ and the covariates $\{\bZ_i =
(A_i,S_i,N_i,P_i,I_i)^{\top}:i=1,\dots,n\}$ using the Cox Proportional Hazard
model
\begin{equation}
\label{eq:CoxPH}
\lambda_i(t) = \lambda_0(t)\exp\{\bbeta^{\top}\bZ_i\},
\end{equation}
where, for the $i-$th subject, $A_i=$age, $S_i=0$ or $1$ if the subject is
female or male, respectively, $N_i=$Neuro (ANC), $P_i=$Platelets and $I_i=0$ or
$1$ if the subject is from the IMRAW cohort or treated at NIH, respectively,
$\lambda_0(t)$ is the continuous baseline hazard rate and $\bbeta =
(\beta_1,\dots,\beta_5)^{\top}$ is the vector of parameters. We assume that the
uncensored time-to-death $T_i^{*}$ and the censoring time $C_i$ are independent,
the subjects are independent and $C_i$ has density $g(t)$ and cumulative
distribution function $G(t)$.

\section{MLE for unknown form baseline $\lambda_{0}(\cdot)$}

Suppose that we have a sample of $n$ independent uncensored survival times
$\{T_1,\dots,T_n\}$, such that, for each $i = 1,\dots,n$, $T_i$ has the
exponential distribution with unknown rate parameter $\lambda$.
The conditional survival function
\begin{align*}
  \cS{t} &= P(T_i^{*}>t\mid Z_i)\\
         &= \exp \left\{ -\int_0^t \lambda_i(s)ds \right\}\\
  &= \exp \left\{ -\exp\{\bbeta^{\top}\bZ_i\}\int_0^t \lambda_0(s)ds \right\}\\
  &= \exp \left\{ -\Lambda_0(t) \exp\{\bbeta^{\top}\bZ_i\} \right\}\\
\end{align*}

By the description, $T_i^{*}$ have density $\cf{\cdot}$ and survival function $\cS{\cdot}$, $C_i$
have density $g(\cdot)$ and survival function $G(\cdot)$.

For subject $i$:

For $\delta_i=0$,
\begin{equation*}
P(T_i=t,\delta_i=0) = P(T_i^{*}>t, C_i=t)=P(T_i^{*}>t)P(C_i=t) = \cS{t}g(t),
\end{equation*}
where the event $\left\{ C_i=t \right\}$ is interpreted as $C_i\in (t,t+dt]$.

For $\delta_i=1$,
\begin{equation*}
P(T_i=t, \delta_i=1) = P(T_i^{*}=t, C_i>t) = P(T_i^{*}=t, C_i>t) = \cf{t}G_{\gamma}(t),
\end{equation*}
where the event $\left\{ T_i^{*}=t \right\}$ is interpreted as $T_i^{*}\in (t,t+dt]$.

So the joint density for subject $i$ is
\begin{equation*}
\left( \cS{t_i}g(t_i) \right)^{1-\delta_i}\left( \cf{t_i}G(t_i) \right)^{\delta_i}.
\end{equation*}
It follows that the full joint likelihood of the observations
$\{T_i,\delta_i,\bZ_i:i=1,\dots,n\}$ is
\begin{align*}
   &\prod_{i=1}^n \left( \cS{T_i}g(T_i) \right)^{1-\delta_i}\left( \cf{T_i}G(T_i) \right)^{\delta_i}\\
  =& \prod_{i=1}^n \left\{ \cS{T_i}^{1-\delta_i}\cf{T_i}^{\delta_i} \right\} \prod_{i=1}^n \left\{ g(T_i)^{1-\delta_i} G(T_i)^{\delta_i} \right\}\\
  =& \prod_{i=1}^n \left\{ \cS{T_i}\clamb{T_i}^{\delta_i} \right\} \prod_{i=1}^n \left\{ g(T_i)^{1-\delta_i} G(T_i)^{\delta_i} \right\}\\
  :=& L(\lambda_0(\cdot),\bbeta;\bT,\bdelta,\bZ) L(g(\cdot);\bT,\bdelta),
\end{align*}
where the first part is the likelihood involves $\lambda_0(\cdot)$ and $\bbeta$:
\begin{align*}
L(\lambda_0(\cdot),\bbeta;\bT,\bdelta,\bZ) &= \prod_{i=1}^n \left\{ \cS{T_i}\clamb{T_i}^{\delta_i} \right\}\\
  &= \prod_{i=1}^n \left\{\exp \left\{ -\Lambda_0(T_i) \exp\{\bbeta^{\top}\bZ_i\} \right\} \left[ \lambda_0(T_i)\exp\{\bbeta^{\top}\bZ_i\} \right]^{\delta_i} \right\},
  \numit\label{eq:Lik1}
\end{align*}
where $\Lambda_0(t) = \int_0^t\lambda_0(s)ds$ is the cumulative hazard funtion.

\section{MLE for known form baseline $\lambda_{0}(\cdot)$}
Suppose the baseline density $f_0(\cdot)$, corresponding to baseline hazard rate
$\lambda_0(\cdot)$, has the expression $f_0(t)=abt^{a-1}\exp \{-bt^a\}$ for
$t\geq 0$ and some unknown parameter $a>0$ and $b>0$.
Therefore, $S_0(t) = \int_t^{\infty}f_0(s)ds = \exp\{-bt^a\} = \exp\{-\Lambda_0(t)\}$.
So $\lambda_0(t) = \frac{d}{dt}\Lambda_0(t)= abt^{a-1}$ for $t>0$.

Then,
\begin{equation}
\label{eq:lam2}
\lambda_i(t) = \lambda_0(t)\exp\{\bbeta^{\top}\bZ_i\} = abt^{a-1}\exp\{\bbeta^{\top}\bZ_i\}.
\end{equation}

Also,
\begin{equation}
\label{eq:S2}
\cS{t} = \exp \left\{ -\Lambda_0(t) \exp\{\bbeta^{\top}\bZ_i\} \right\} = \exp \left\{ -bt^a \exp\{\bbeta^{\top}\bZ_i\} \right\}.
\end{equation}
By \eqref{eq:Lik1}, we have
\begin{align*}
L(\lambda_0(\cdot),\bbeta;\bT,\bdelta,\bZ) &= \prod_{i=1}^n \left\{\exp \left\{ -\Lambda_0(T_i) \exp\{\bbeta^{\top}\bZ_i\} \right\} \left[ \lambda_0(T_i)\exp\{\bbeta^{\top}\bZ_i\} \right]^{\delta_i} \right\}\\
  &= \prod_{i=1}^n \left\{\exp \left\{ -bT_i^a \exp\{\bbeta^{\top}\bZ_i\} \right\} \left[ abT_i^{a-1}\exp\{\bbeta^{\top}\bZ_i\} \right]^{\delta_i} \right\}
  \numit\label{eq:Lik2}
\end{align*}
\section{Application on IMRAW-IST dataset}
\begin{enumerate}
\item We would like to estimate the effects of age, sex, ANC, platelets, and IST
treatment on the overall survival distributions using this IMRAW-IST dataset and
the Cox model \eqref{eq:CoxPH} with the assumption that the baseline hazard rate
$\lambda_0(t)$ does not have a known parametric form.

\item Suppose that we suspect that the effects of IST on the overall survival
  time distributions may be different for men and women after adjusting for age,
  ANC and platelets, while the effects of age, ANC and platelets may be entered
  as linear terms in \eqref{eq:CoxPH}.

  
\end{enumerate}


\appendix

\section{R Code}

Code is avaliable on \url{https://github.com/mr0112358/SurvivalAnalysis-IMRAW-IST/blob/master/Assignment4.R}

\begin{lstlisting}

\end{lstlisting}

\end{document}
