\documentclass[11pt]{article}

\usepackage{amsmath, amssymb, amsthm, bm, bbm,graphicx, mathtools, enumerate,multirow}
\usepackage[letterpaper, left=1.1truein, right=1.1truein, top = 1.1truein,
bottom = 1.1truein]{geometry}
\usepackage[affil-it]{authblk}
\usepackage{natbib}
\usepackage{hyperref}
\usepackage[usenames,dvipsnames]{color}
\usepackage[ruled, vlined, lined, commentsnumbered]{algorithm2e}
\usepackage{prettyref,soul}
\usepackage{float}
\usepackage{setspace}
\usepackage{listings}

\lstset{ 
  language=R,                     % the language of the code
  basicstyle=\tiny\ttfamily, % the size of the fonts that are used for the code
  numbers=left,                   % where to put the line-numbers
  numberstyle=\tiny\color{Blue},  % the style that is used for the line-numbers
  stepnumber=1,                   % the step between two line-numbers. If it is 1, each line
                                  % will be numbered
  numbersep=5pt,                  % how far the line-numbers are from the code
  backgroundcolor=\color{white},  % choose the background color. You must add \usepackage{color}
  showspaces=false,               % show spaces adding particular underscores
  showstringspaces=false,         % underline spaces within strings
  showtabs=false,                 % show tabs within strings adding particular underscores
  frame=single,                   % adds a frame around the code
  rulecolor=\color{black},        % if not set, the frame-color may be changed on line-breaks within not-black text (e.g. commens (green here))
  tabsize=2,                      % sets default tabsize to 2 spaces
  captionpos=b,                   % sets the caption-position to bottom
  breaklines=true,                % sets automatic line breaking
  breakatwhitespace=false,        % sets if automatic breaks should only happen at whitespace
  keywordstyle=\color{RoyalBlue},      % keyword style
  commentstyle=\color{YellowGreen},   % comment style
  stringstyle=\color{ForestGreen}      % string literal style
} 

\newcommand{\numit}{\stepcounter{equation}\tag{\theequation}}


\DeclareMathOperator*{\argmin}{arg\,min}
\newcommand{\E}{\mbox{{\rm E}}}
\newcommand{\Var}{\mbox{{\rm Var}}}
\newcommand{\tr}{\mbox{{\rm tr}}}
\newcommand{\diag}{\mbox{{\rm diag}}}

\newtheorem{thm}{Theorem}
\newtheorem{defi}{Definition}
\newtheorem{lem}{Lemma}
\newtheorem{coro}{Corollary}
\newtheorem{prop}{Proposition}
\newtheorem{ex}{Example}
\newtheorem{rmk}{Remark}
\newtheorem{asmp}{Assumption}

\newcommand{\hS}{\hat{S}}
\newcommand{\hsS}{\hat{S}^{*}}
\newcommand{\hlam}{\hat{\lambda}}
\newcommand{\hLam}{\hat{\Lambda}}
\newcommand{\hsLam}{\hat{\Lambda}^{*}}
\newcommand{\hF}{\hat{F}}
\newcommand{\tsF}{{\tilde{F}^{*}}}
\newcommand{\invtsF}{{\tilde{F}^{*-1}}}


\begin{document}

\title{STAT 6227, Assignment \#2}
%\author{Rui Miao}

\maketitle

\section{Description of Nonparametric Estimators and Bootstrap CIs}
For subject $i$, let $T_i$ be the time of event of interest and $C_i$ be the
censoring time. In practice, we only observe i.i.d. data $\left\{ (X_i,
  \Delta_i) \right\}_{i=1}^n$, where $X_i = T_i\wedge C_i$ is the observation time
and $\Delta_i = I(T_i<C_i)$ is the indicator of event. Assume $T_i$ is
independent of $C_i$ for all $i$.

\subsection{Kaplan-Meier (KM) Estimators of $S(t)$}
Suppose the observed failure times are $t_1<t_2<\dots < t_{n_D}$, where $n_D$ is
the number of unique times at which deaths are observed. The Kaplan-Meier
estimator of survival function $S(t) = P(X>t)$ is given by:

\begin{equation}
\label{eq:KM}
\hS (t) = \prod_{j:t_j\leq t} \left\{ 1-\frac{D_j}{Y_j} \right\},
\end{equation}

where $Y_j$ is the number ``at risk'' at $t=t_j$ and $D_j$ is the number of
failures at $t = t_j$.

Note that the KM estimator $\hS(t)$ is well-defined up to the last observation
time, $\tau = \max \{X_1,\dots,X_n\}$. If the last observation is censored,
$\hS(t)$ will not go right down to 0.

\subsection{Nelson-Aalen (NA) Estimators of $\Lambda (t)$}

To estimate cumulative hazard function $\Lambda (t) = \int_0^t \lambda (s) ds =
-\log S(t)$,
taking $\hlam (t_j) = D_j/Y_j$, the Nelson-Aalen estimator is

\begin{equation}
\label{eq:NA}
\hLam (t) = \sum_{j:t_j\leq t} \frac{D_j}{Y_j}.
\end{equation}

\subsection{Bootstrap Confidence Intervals}
Steps for pointwise bootstrap confidence intervals:
Based on observed data $\left\{ (x_i,\delta_i) \right\}_{i=1}^n$ and a specific
time point $t$:
\begin{enumerate}[(a)]
\item Draw a bootstrap sample $\left\{ (X_i^{*},\Delta_i^{*}) \right\}_{i=1}^n$
  by independent sampling $n$ times with replacement from $\hF$, the empirical
  distribution putting mass $1/n$ at each point $(x_i,\delta_i)$;
\item Based on the bootstrap sample $\left\{ (X_i^{*},\Delta_i^{*})
  \right\}_{i=1}^n$, calculate $\hsS (t)$ and $\hsLam (t)$;
\item Independently repeat above steps for $B$ times, obtaining
  $\hsS_1(t),\dots, \hsS_B(t)$ and $\hsLam_1(t),\dots,\hsLam_B(t)$.
\item According to the linear interpolated empirical distribution $\tsF_S$ of
  $\hsS_1(t),\dots, \hsS_B(t)$ or $\tsF_{\Lambda}$ of
  $\hsLam_1(t),\dots,\hsLam_B(t)$, the $1-\alpha$ bootstrap CI for $S(t)$ is

\begin{equation*}
  \left[\invtsF_S\left(\frac{\alpha}{2}\right), \invtsF_S\left(1-\frac{\alpha}{2}\right)\right],
\end{equation*}

  for $\Lambda (t)$ is 
  
\begin{equation*}
\left[\invtsF_{\Lambda}\left(\frac{\alpha}{2}\right), \invtsF_{\Lambda}\left(1-\frac{\alpha}{2}\right)\right],
\end{equation*}

where $\alpha = 0.05$ for $95\%$ CIs.

\end{enumerate}

\section{Application to the Survival Functions of the IMRAW-IST data}

\begin{table}[H]
\caption{KM estimates of $S(1)$, $S(2)$, $S(5)$
             with 95\% CIs by Greenwood formula and bootstrap for IMRAW patients} 
\centering
\begin{tabular}{rrrr}
  \hline
  \hline
 & $\hS(t)$ & [2.5\%, 97.5\%] & Boot [2.5\%, 97.5\%] \\ 
  \hline
  year 1 & 0.7395 & [0.7093, 0.7709] & [0.7088, 0.7663] \\ 
  year 3 & 0.4960 & [0.4605, 0.5342] & [0.4617, 0.5309] \\ 
  year 5 & 0.3413 & [0.3054, 0.3815] & [0.2977, 0.3782] \\ 
   \hline
\end{tabular}
\end{table}

\begin{table}[H]
\caption{KM estimates of $S(1)$, $S(2)$, $S(5)$
             with 95\% CIs by Greenwood formula and bootstrap for IST patients} 
\centering
\begin{tabular}{rrrr}
  \hline
  \hline
 & $\hS(t)$ & [2.5\%, 97.5\%] & Boot [2.5\%, 97.5\%] \\ 
  \hline
  year 1 & 0.7864 & [0.7181, 0.8613] & [0.7192, 0.8560] \\ 
  year 3 & 0.5837 & [0.5032, 0.6771] & [0.5005, 0.6695] \\ 
  year 5 & 0.5665 & [0.4856, 0.6609] & [0.4886, 0.6565] \\ 
   \hline
\end{tabular}
\end{table}

\section{Application to the Cumulative Harzard Functions of the IMRAW-IST data}

\begin{table}[H]
\caption{NA estimates of $\Lambda(1)$, $\Lambda(2)$, $\Lambda(5)$ 
             with 95\% CIs by asymptotics and bootstrap for IMRAW patients} 
\centering
\begin{tabular}{rrrr}
  \hline
  \hline
 & $\hLam(t)$ & [2.5\%, 97.5\%] & Boot [2.5\%, 97.5\%] \\ 
  \hline
  year 1 & 0.3013 & [0.2598, 0.3428] & [0.2655, 0.3432] \\ 
  year 3 & 0.6998 & [0.6257, 0.7739] & [0.6316, 0.7703] \\ 
  year 5 & 1.0723 & [0.9613, 1.1832] & [0.9690, 1.2057] \\ 
   \hline
\end{tabular}
\end{table}

\begin{table}[H]
\caption{NA estimates of $\Lambda(1)$, $\Lambda(2)$, $\Lambda(5)$ 
             with 95\% CIs by asymptotics and bootstrap for IST patients} 
\centering
\begin{tabular}{rrrr}
  \hline
  \hline
 & $\hLam(t)$ & [2.5\%, 97.5\%] & Boot [2.5\%, 97.5\%] \\ 
  \hline
  year 1 & 0.2391 & [0.1486, 0.3296] & [0.1543, 0.3265] \\ 
  year 3 & 0.5352 & [0.3877, 0.6827] & [0.3978, 0.6835] \\ 
  year 5 & 0.5649 & [0.4118, 0.7181] & [0.4170, 0.7071] \\ 
   \hline
\end{tabular}
\end{table}

\appendix

\section{R Code}

Code is avaliable on 

\begin{lstlisting}
# Read Data
library(doParallel)
library(survival)
library(xtable)
IMRAWIST = data.frame(readxl::read_excel("IMRAWandISTnov.xls"))
IMRAW = subset(IMRAWIST, NIH == 0)
IST = subset(IMRAWIST, NIH == 1)
IMRAWIST = list(IMRAW, IST)

registerDoParallel(detectCores())

# KM estimator for survival functions
KMtables = foreach(d = 1:2) %dopar% {
  dataName = ifelse(d == 1, "IMRAW", "IST")
  
  KM_death = survfit(
    Surv(time = survival, event = DIED) ~ 1,
    type = "kaplan-meier",
    error = "greenwood",
    data = IMRAWIST[[d]]
  )
  
  SurvBoot = foreach(B = 1:500, .combine = 'rbind') %dopar% {
    n = nrow(IMRAWIST[[d]])
    set.seed(B)
    bootIndex = sample.int(n, replace = TRUE)
    KM_death_boot = survfit(Surv(time = survival, event = DIED) ~ 1,
                            type = "kaplan-meier",
                            data = IMRAWIST[[d]][bootIndex, ])
    KM_death_boot = summary(KM_death_boot, times = c(1, 3, 5), scale = 365.25)
    KM_death_boot$surv
  }
  
  KM_death = summary(KM_death, times = c(1, 3, 5), scale = 365.25)
  
  TableKM = t(rbind(
    KM_death$surv,
    KM_death$lower,
    KM_death$upper,
    bootCI = sapply(1:3, function(y) {
      quantile(SurvBoot[, y], probs = c(0.025, 0.975))
    })
  ))
  rownames(TableKM) = c("year 1", "year 3", "year 5")
  colnames(TableKM) = c("$\\hS(t)$", "2.5%", "97.5%", "2.5% Boot", "97.5% Boot")
  xtable(
    TableKM,
    digits = 4,
    caption = paste0(
      "KM estimates of $S(1)$, $S(2)$, $S(5)$
             with 95% CIs by Greenwood formula and bootstrap for ",
      dataName,
      " patients"
    )
  )
}


# NA estimator for cumulative hazard function
NAtables = foreach(d = 1:2) %dopar% {
  dataName = ifelse(d == 1, "IMRAW", "IST")
  
  NA_death = survfit(Surv(time = survival, event = DIED) ~ 1,
                     type = "fleming-harrington",
                     data = IMRAWIST[[d]])
  
  CumHazardBoot = foreach(B = 1:500, .combine = 'rbind') %dopar% {
    n = nrow(IMRAWIST[[d]])
    set.seed(B)
    bootIndex = sample.int(n, replace = TRUE)
    NA_death_boot = survfit(Surv(time = survival, event = DIED) ~ 1,
                            type = "fleming-harrington",
                            data = IMRAWIST[[d]][bootIndex, ])
    NA_death_boot = summary(NA_death_boot, times = c(1, 3, 5), scale = 365.25)
    NA_death_boot$cumhaz
  }
  
  NA_death = summary(NA_death, times = c(1, 3, 5), scale = 365.25)
  
  TableNA = t(rbind(
    NA_death$cumhaz,
    -log(NA_death$upper),
    -log(NA_death$lower),
    bootCI = sapply(1:3, function(y) {
      quantile(CumHazardBoot[, y], probs = c(0.025, 0.975))
    })
  ))
  rownames(TableNA) = c("year 1", "year 3", "year 5")
  colnames(TableNA) = c("$\\hLam(t)$", "2.5%", "97.5%", "2.5% Boot", "97.5% Boot")
  xtable(
    TableNA,
    digits = 4,
    caption = paste0(
      "NA estimates of $\\Lambda(1)$, $\\Lambda(2)$, $\\Lambda(5)$
             with 95% CIs by asymptotics and bootstrap for ",
      dataName,
      " patients"
    )
  )
}

print(KMtables[[1]])
print(KMtables[[2]])
print(NAtables[[1]])
print(NAtables[[2]])

\end{lstlisting}

\end{document}
