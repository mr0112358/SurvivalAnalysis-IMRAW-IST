\documentclass[11pt]{article}

\usepackage{amsmath, amssymb, amsthm, bm, bbm,graphicx, mathtools, enumerate,multirow}
\usepackage[letterpaper, left=1.1truein, right=1.1truein, top = 1.1truein,
bottom = 1.1truein]{geometry}
\usepackage[affil-it]{authblk}
\usepackage{natbib}
\usepackage{hyperref}
\usepackage[usenames,dvipsnames]{color}
\usepackage[ruled, vlined, lined, commentsnumbered]{algorithm2e}
\usepackage{prettyref,soul}
\usepackage{float}
\usepackage{setspace}
\usepackage{fancyvrb}
\usepackage{listings}

\lstset{ 
  language=R,                     % the language of the code
  basicstyle=\small\ttfamily, % the size of the fonts that are used for the code
  numbers=left,                   % where to put the line-numbers
  numberstyle=\small\color{Blue},  % the style that is used for the line-numbers
  stepnumber=1,                   % the step between two line-numbers. If it is 1, each line
                                  % will be numbered
  numbersep=5pt,                  % how far the line-numbers are from the code
  backgroundcolor=\color{white},  % choose the background color. You must add \usepackage{color}
  showspaces=false,               % show spaces adding particular underscores
  showstringspaces=false,         % underline spaces within strings
  showtabs=false,                 % show tabs within strings adding particular underscores
  frame=single,                   % adds a frame around the code
  rulecolor=\color{black},        % if not set, the frame-color may be changed on line-breaks within not-black text (e.g. commens (green here))
  tabsize=2,                      % sets default tabsize to 2 spaces
  captionpos=b,                   % sets the caption-position to bottom
  breaklines=true,                % sets automatic line breaking
  breakatwhitespace=false,        % sets if automatic breaks should only happen at whitespace
  keywordstyle=\color{RoyalBlue},      % keyword style
  commentstyle=\color{YellowGreen},   % comment style
  stringstyle=\color{ForestGreen}      % string literal style
} 

\newcommand{\numit}{\stepcounter{equation}\tag{\theequation}}


\DeclareMathOperator*{\argmin}{arg\,min}
\newcommand{\E}{\mbox{{\rm E}}}
\newcommand{\Var}{\mbox{{\rm Var}}}
\newcommand{\tr}{\mbox{{\rm tr}}}
\newcommand{\diag}{\mbox{{\rm diag}}}

\newtheorem{thm}{Theorem}
\newtheorem{defi}{Definition}
\newtheorem{lem}{Lemma}
\newtheorem{coro}{Corollary}
\newtheorem{prop}{Proposition}
\newtheorem{ex}{Example}
\newtheorem{rmk}{Remark}
\newtheorem{asmp}{Assumption}

\newcommand{\hS}{\hat{S}}
\newcommand{\hsS}{\hat{S}^{*}}
\newcommand{\hlam}{\hat{\lambda}}
\newcommand{\hLam}{\hat{\Lambda}}
\newcommand{\htheta}{\hat{\theta}}

\newcommand{\cS}[1]{S \left( #1 \mid \bZ_i \right)}
\newcommand{\cf}[1]{f \left( #1 \mid \bZ_i \right)}
\newcommand{\clamb}[1]{\lambda \left( #1 \mid \bZ_i \right)}

\newcommand{\bT}{{\bf{T}}}
\newcommand{\bX}{{\bf{X}}}
\newcommand{\bZ}{{\bf{Z}}}
\newcommand{\bdelta}{{\bm \delta}}
\newcommand{\bbeta}{{\bm \beta}}
\newcommand{\bgamma}{{\bm \gamma}}



\begin{document}

\title{STAT 6227, Assignment \#5}
\maketitle

\section{Description}
As an alternative to HW\#4, we woule like to consider some parametric approaches
to analyze the IMRAW-IST dataset studied in Sloand et al., (Jouranal of
Clinical Oncology, 2008, Vol. 26, No. 15, 2505-2511) and would like to model the
hazard rate for time-to-death $T$ based on the observed censored data
$\{(T_i,\delta_i):i=1,\dots,n\}$ and the covariates $\{\bZ_i =
(A_i,S_i,N_i,P_i,I_i)^{\top}:i=1,\dots,n\}$ using the Cox Proportional Hazard
model
\begin{equation}
\label{eq:CoxPH}
\lambda_i(t) = \lambda_0(t)\exp\{\bbeta^{\top}\bZ_i\},
\end{equation}
where, for the $i-$th subject, $A_i=$age, $S_i=0$ or $1$ if the subject is
female or male, respectively, $N_i=$Neuro (ANC), $P_i=$Platelets and $I_i=0$ or
$1$ if the subject is from the IMRAW cohort or treated at NIH, respectively,
$\lambda_0(t)$ is the continuous baseline hazard rate and $\bbeta =
(\beta_1,\dots,\beta_5)^{\top}$ is the vector of parameters.
In order to define a clinically meaningful $\lambda_0(\cdot)$, we define $A_i$,
$N_i$ and $P_i$ to be the ``centered'' version of age, ANC and platelets, i.e.,
they are obtained by substracting their mean values from their actual values.
We assume that the uncensored time-to-death $T_i^{*}$ and the censoring time $C_i$ are independent,
the subjects are independent and $C_i$ has density $g(t)$ and cumulative
distribution function $G(t)$.

\section{Survival and likelihood function for different parametric forms of baseline hazard rate $\lambda_{0}(\cdot)$}
By definition of survival function, 
\begin{equation}
\label{eq:SurvFunc}
\cS{t} = \exp \left\{ -\Lambda_0(t) \exp\{\bbeta^{\top}\bZ_i\} \right\}.
\end{equation}
where $\Lambda_0(t) = \int_0^t \lambda_0(s)ds$ is the baseline cumulative hazard function.

From HW \#4, we know that the joint likelihood function of the observations
$\{T_i, \delta_i,\bZ_i:i=1,\dots,n\}$ as a function of $\lambda_0(\cdot)$,
$\bbeta$ and $\{T_i, \delta_i,\bZ_i:i=1,\dots,n\}$ is
\begin{align*}
L(\lambda_0(\cdot),\bbeta;\bT,\bdelta,\bZ) &= \prod_{i=1}^n \left\{ \cS{T_i}\clamb{T_i}^{\delta_i} \right\}\\
  &= \prod_{i=1}^n \left\{\exp \left\{ -\Lambda_0(T_i) \exp\{\bbeta^{\top}\bZ_i\} \right\} \left[ \lambda_0(T_i)\exp\{\bbeta^{\top}\bZ_i\} \right]^{\delta_i} \right\}.
  \numit\label{eq:Lik}
\end{align*}

We consider three parametric assumptions:

\begin{enumerate}
\item When $\lambda_0(\cdot)$ is hazard rate of $\text{Exp}(\gamma)$, 
  $\lambda_0(t)=\gamma$ and $\Lambda_0(t)=\gamma t$ in \eqref{eq:SurvFunc} and \eqref{eq:Lik};
\item When $\lambda_0(\cdot)$ is hazard rate of $\text{Weibull}(\bgamma)$ with
  $\bgamma=(\gamma_1,\gamma_2)^{\top}$, $\lambda_0(t)=\gamma_1\gamma_2
  t^{\gamma_1 - 1}$ and $\Lambda_0(t) = \gamma_2 t^{\gamma_1-1}$ in \eqref{eq:SurvFunc} and \eqref{eq:Lik};
\item When $\lambda_0(\cdot)$ is hazard rate of
  $\text{log-Normal}(\mu,\sigma^2)$, $\Lambda_0(t) = -\log \left\{ 1- \Phi
    \left( \frac{\log t - \mu}{\sigma} \right)\right\}$ and $\lambda_0(t) =
  \frac{d}{dt}\Lambda_0(t) = \frac{\exp \left\{ -\frac{1}{2} \left( \frac{\log t
      - \mu}{\sigma}^2 \right)\right\}}{\sqrt{2\pi}\sigma t \left\{ 1-\Phi \left(
      \frac{\log t - \mu}{\sigma} \right) \right\}}$ in \eqref{eq:SurvFunc} and \eqref{eq:Lik}.
\end{enumerate}

\section{Application on IMRAW-IST dataset}
We estimate parameters of $\bbeta$ with p-values and corresponding estimates of
hazard ratios with 95\% confidence intervals for three parametric assumptions
aforementioned. The results are summarized in Table 1.



\begin{enumerate}
\item We would like to estimate the effects of age, sex, ANC, platelets, and IST
treatment on the overall survival distributions using this IMRAW-IST dataset and
the Cox model \eqref{eq:CoxPH} with the assumption that the baseline hazard rate
$\lambda_0(t)$ does not have a known parametric form.

\begin{lstlisting}
# Read Data
library(survival)
IMRAWIST = data.frame(readxl::read_excel("IMRAWandISTnov.xls"))

# Cox Model without interaction of treatment and sex
coxfit1<-coxph(Surv(time = survival, event = DIED, type = "right") 
                ~ AGE + I(GENDER=="M") + NEUTRO + PLATE + NIH,
                data = IMRAWIST)
summary(coxfit1)
\end{lstlisting}

\begin{Verbatim}[fontsize=\small]
Call:
coxph(formula = Surv(time = survival, event = DIED, type = "right") ~ 
    AGE + I(GENDER == "M") + NEUTRO + PLATE + NIH, data = IMRAWIST)

  n= 917, number of events= 554 
   (28 observations deleted due to missingness)

                           coef  exp(coef)   se(coef)      z Pr(>|z|)    
AGE                   0.0313555  1.0318523  0.0038434  8.158 3.40e-16 ***
I(GENDER == "M")TRUE  0.3172310  1.3733198  0.0900924  3.521 0.000430 ***
NEUTRO                0.0292087  1.0296395  0.0161931  1.804 0.071266 .  
PLATE                -0.0022129  0.9977895  0.0003635 -6.088 1.14e-09 ***
NIH                  -0.5570141  0.5729172  0.1565176 -3.559 0.000373 ***
---
Signif. codes:  0 '***' 0.001 '**' 0.01 '*' 0.05 '.' 0.1 ' ' 1

                     exp(coef) exp(-coef) lower .95 upper .95
AGE                     1.0319     0.9691    1.0241    1.0397
I(GENDER == "M")TRUE    1.3733     0.7282    1.1510    1.6385
NEUTRO                  1.0296     0.9712    0.9975    1.0628
PLATE                   0.9978     1.0022    0.9971    0.9985
NIH                     0.5729     1.7455    0.4216    0.7786

Concordance= 0.644  (se = 0.013 )
Likelihood ratio test= 165.1  on 5 df,   p=<2e-16
Wald test            = 141.7  on 5 df,   p=<2e-16
Score (logrank) test = 144.3  on 5 df,   p=<2e-16
\end{Verbatim}

The estimates of $\bbeta$ and the corresponding hazard ratios are given above 
in columns \texttt{coef} and \texttt{exp(coef)} respectively. After adjusting
age, sex, ANC and platelets, the hazard ratio of IST treatment is 0.5927
(p-value 0.000373). It shows that IST treatment can significantly lower down
hazard of death.

\item Suppose that we suspect that the effects of IST on the overall survival
  time distributions may be different for men and women after adjusting for age,
  ANC and platelets, while the effects of age, ANC and platelets may be entered
  as linear terms in \eqref{eq:CoxPH}.

  To take the potential different effects of IST into account for different sex,
  we add the interaction term of IST:sex. We adjust
  $\{\bZ_i=(A_i,S_i,N_i,P_i,I_i, S_iI_i)^{\top}:i=1,\dots,n\}$ and $\bbeta =
  (\beta_1,\dots,\beta_6)$ in model \eqref{eq:CoxPH}. Then $e^{\beta_6}$ is the
  hazard ratio of the IST effect for male over that for female, after
  adjusting for age, ANC and platelets.

  The hypotheses for testing the possible different IST effects are
\begin{equation}
\label{eq:hypothesis}
H_0: \beta_6=0\quad\text{v.s.}\quad H_1:\beta_6\neq 0.
\end{equation}

\begin{lstlisting}
# Cox Model with interaction of treatment and sex
coxfit2<-coxph(Surv(time = survival, event = DIED, type = "right") 
                 ~ AGE + I(GENDER=="M") + NEUTRO + PLATE + NIH 
                   + I(GENDER=="M"):NIH,
                 data = IMRAWIST)
summary(coxfit2)
\end{lstlisting}

\begin{Verbatim}[fontsize=\small]
Call:
coxph(formula = Surv(time = survival, event = DIED, type = "right") ~ 
    AGE + I(GENDER == "M") + NEUTRO + PLATE + NIH + I(GENDER == 
        "M"):NIH, data = IMRAWIST)

  n= 917, number of events= 554 
   (28 observations deleted due to missingness)

                               coef  exp(coef)   se(coef)      z Pr(>|z|)
AGE                       0.0310526  1.0315397  0.0038508  8.064 7.39e-16 ***
I(GENDER == "M")TRUE      0.2861726  1.3313222  0.0936597  3.055  0.00225 **
NEUTRO                    0.0290407  1.0294664  0.0162387  1.788  0.07372 .
PLATE                    -0.0022068  0.9977957  0.0003631 -6.078 1.22e-09 ***
NIH                      -0.8649707  0.4210639  0.3201946 -2.701  0.00691 **
I(GENDER == "M")TRUE:NIH  0.4039306  1.4977000  0.3527289  1.145  0.25214    
---
Signif. codes:  0 '***' 0.001 '**' 0.01 '*' 0.05 '.' 0.1 ' ' 1

                         exp(coef) exp(-coef) lower .95 upper .95
AGE                         1.0315     0.9694    1.0238    1.0394
I(GENDER == "M")TRUE        1.3313     0.7511    1.1081    1.5996
NEUTRO                      1.0295     0.9714    0.9972    1.0628
PLATE                       0.9978     1.0022    0.9971    0.9985
NIH                         0.4211     2.3749    0.2248    0.7887
I(GENDER == "M")TRUE:NIH    1.4977     0.6677    0.7502    2.9900

Concordance= 0.644  (se = 0.013 )
Likelihood ratio test= 166.5  on 6 df,   p=<2e-16
Wald test            = 139.3  on 6 df,   p=<2e-16
Score (logrank) test = 144.3  on 6 df,   p=<2e-16
\end{Verbatim}
Although the estimated hazard ratio $e^{\hat\beta_6}=1.4977$, meaning IST for
male is not as good as female, the p-value 0.25214 does not support our
conclusion. We failed to reject null hypothesis $H_0$. There is no significant
difference of IST effects between male and female.
\end{enumerate}


\appendix

\section{R Code}

Code is avaliable on \url{https://github.com/mr0112358/SurvivalAnalysis-IMRAW-IST/blob/master/Assignment4.R}


\end{document}
