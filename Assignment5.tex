\documentclass[11pt]{article}

\usepackage{amsmath, amssymb, amsthm, bm, bbm,graphicx, mathtools, enumerate,multirow}
\usepackage[letterpaper, left=1.1truein, right=1.1truein, top = 1.1truein,
bottom = 1.1truein]{geometry}
\usepackage[affil-it]{authblk}
\usepackage{natbib}
\usepackage{hyperref}
\usepackage[usenames,dvipsnames]{color}
\usepackage[ruled, vlined, lined, commentsnumbered]{algorithm2e}
\usepackage{prettyref,soul}
\usepackage{float}
\usepackage{setspace}
\usepackage{fancyvrb}
\usepackage{listings}

\lstset{ 
  language=R,                     % the language of the code
  basicstyle=\tiny\ttfamily, % the size of the fonts that are used for the code
  numbers=left,                   % where to put the line-numbers
  numberstyle=\tiny\color{Blue},  % the style that is used for the line-numbers
  stepnumber=1,                   % the step between two line-numbers. If it is 1, each line
                                  % will be numbered
  numbersep=5pt,                  % how far the line-numbers are from the code
  backgroundcolor=\color{white},  % choose the background color. You must add \usepackage{color}
  showspaces=false,               % show spaces adding particular underscores
  showstringspaces=false,         % underline spaces within strings
  showtabs=false,                 % show tabs within strings adding particular underscores
  frame=single,                   % adds a frame around the code
  rulecolor=\color{black},        % if not set, the frame-color may be changed on line-breaks within not-black text (e.g. commens (green here))
  tabsize=2,                      % sets default tabsize to 2 spaces
  captionpos=b,                   % sets the caption-position to bottom
  breaklines=true,                % sets automatic line breaking
  breakatwhitespace=false,        % sets if automatic breaks should only happen at whitespace
  keywordstyle=\color{RoyalBlue},      % keyword style
  commentstyle=\color{YellowGreen},   % comment style
  stringstyle=\color{ForestGreen}      % string literal style
} 

\newcommand{\numit}{\stepcounter{equation}\tag{\theequation}}


\DeclareMathOperator*{\argmin}{arg\,min}
\newcommand{\E}{\mbox{{\rm E}}}
\newcommand{\Var}{\mbox{{\rm Var}}}
\newcommand{\tr}{\mbox{{\rm tr}}}
\newcommand{\diag}{\mbox{{\rm diag}}}

\newtheorem{thm}{Theorem}
\newtheorem{defi}{Definition}
\newtheorem{lem}{Lemma}
\newtheorem{coro}{Corollary}
\newtheorem{prop}{Proposition}
\newtheorem{ex}{Example}
\newtheorem{rmk}{Remark}
\newtheorem{asmp}{Assumption}

\newcommand{\hS}{\hat{S}}
\newcommand{\hsS}{\hat{S}^{*}}
\newcommand{\hlam}{\hat{\lambda}}
\newcommand{\hLam}{\hat{\Lambda}}
\newcommand{\htheta}{\hat{\theta}}

\newcommand{\cS}[1]{S \left( #1 \mid \bZ_i \right)}
\newcommand{\cf}[1]{f \left( #1 \mid \bZ_i \right)}
\newcommand{\clamb}[1]{\lambda \left( #1 \mid \bZ_i \right)}

\newcommand{\bT}{{\bf{T}}}
\newcommand{\bX}{{\bf{X}}}
\newcommand{\bZ}{{\bf{Z}}}
\newcommand{\bdelta}{{\bm \delta}}
\newcommand{\bbeta}{{\bm \beta}}
\newcommand{\bgamma}{{\bm \gamma}}



\begin{document}

\title{STAT 6227, Assignment \#5}
\maketitle

\section{Description}
As an alternative to HW\#4, we woule like to consider some parametric approaches
to analyze the IMRAW-IST dataset studied in Sloand et al., (Jouranal of
Clinical Oncology, 2008, Vol. 26, No. 15, 2505-2511) and would like to model the
hazard rate for time-to-death $T$ based on the observed censored data
$\{(T_i,\delta_i):i=1,\dots,n\}$ and the covariates $\{\bZ_i =
(A_i,S_i,N_i,P_i,I_i)^{\top}:i=1,\dots,n\}$ using the Cox Proportional Hazard
model
\begin{equation}
\label{eq:CoxPH}
\lambda_i(t) = \lambda_0(t)\exp\{\bbeta^{\top}\bZ_i\},
\end{equation}
where, for the $i-$th subject, $A_i=$age, $S_i=0$ or $1$ if the subject is
female or male, respectively, $N_i=$Neuro (ANC), $P_i=$Platelets and $I_i=0$ or
$1$ if the subject is from the IMRAW cohort or treated at NIH, respectively,
$\lambda_0(t)$ is the continuous baseline hazard rate and $\bbeta =
(\beta_1,\dots,\beta_5)^{\top}$ is the vector of parameters.
In order to define a clinically meaningful $\lambda_0(\cdot)$, we define $A_i$,
$N_i$ and $P_i$ to be the ``centered'' version of age, Neutro(ANC) and platelets, i.e.,
they are obtained by substracting their mean values from their actual values.
We assume that the uncensored time-to-death $T_i^{*}$ and the censoring time $C_i$ are independent,
the subjects are independent and $C_i$ has density $g(t)$ and cumulative
distribution function $G(t)$.

\section{Survival and likelihood function for different parametric forms of baseline hazard rate $\lambda_{0}(\cdot)$}
By definition of survival function, 
\begin{equation}
\label{eq:SurvFunc}
\cS{t} = \exp \left\{ -\Lambda_0(t) \exp\{\bbeta^{\top}\bZ_i\} \right\}.
\end{equation}
where $\Lambda_0(t) = \int_0^t \lambda_0(s)ds$ is the baseline cumulative hazard function.

From HW \#4, we know that the joint likelihood function of the observations
$\{T_i, \delta_i,\bZ_i:i=1,\dots,n\}$ as a function of $\lambda_0(\cdot)$,
$\bbeta$ and $\{T_i, \delta_i,\bZ_i:i=1,\dots,n\}$ is
\begin{align*}
L(\lambda_0(\cdot),\bbeta;\bT,\bdelta,\bZ) &= \prod_{i=1}^n \left\{ \cS{T_i}\clamb{T_i}^{\delta_i} \right\}\\
  &= \prod_{i=1}^n \left\{\exp \left\{ -\Lambda_0(T_i) \exp\{\bbeta^{\top}\bZ_i\} \right\} \left[ \lambda_0(T_i)\exp\{\bbeta^{\top}\bZ_i\} \right]^{\delta_i} \right\}.
  \numit\label{eq:Lik}
\end{align*}

We consider three parametric assumptions:

\begin{enumerate}
\item When $\lambda_0(\cdot)$ is hazard rate of $\text{Exp}(\gamma)$, 
  $\lambda_0(t)=\gamma$ and $\Lambda_0(t)=\gamma t$ in \eqref{eq:SurvFunc} and \eqref{eq:Lik};
\item When $\lambda_0(\cdot)$ is hazard rate of $\text{Weibull}(\bgamma)$ with
  $\bgamma=(\gamma_1,\gamma_2)^{\top}$, $\lambda_0(t)=\gamma_1\gamma_2
  t^{\gamma_1 - 1}$ and $\Lambda_0(t) = \gamma_2 t^{\gamma_1}$ in \eqref{eq:SurvFunc} and \eqref{eq:Lik};
\item When $\lambda_0(\cdot)$ is hazard rate of
  $\text{log-Normal}(\mu,\sigma^2)$, $\Lambda_0(t) = -\log \left\{ 1- \Phi
    \left( \frac{\log t - \mu}{\sigma} \right)\right\}$ and $\lambda_0(t) =
  \frac{d}{dt}\Lambda_0(t) = \frac{\exp \left\{ -\frac{1}{2} \left( \frac{\log t
      - \mu}{\sigma}^2 \right)\right\}}{\sqrt{2\pi}\sigma t \left\{ 1-\Phi \left(
      \frac{\log t - \mu}{\sigma} \right) \right\}}$ in \eqref{eq:SurvFunc} and \eqref{eq:Lik}.
\end{enumerate}

\section{Application on IMRAW-IST dataset}
We estimate parameters of $\bbeta$ with p-values and corresponding estimates of
hazard ratios with 95\% confidence intervals for three parametric assumptions
aforementioned.

The hazard ratio (HR) for $\beta_i$ is given by $\psi_i = \exp
\{-\beta_i/\sigma\}$, $i=1,\dots,5$, where the scale parameter $\sigma=1$ for Exponential baseline,
$\sigma=1/\gamma_1$ for Weibull$(\gamma_1,\gamma_2)$ baseline and
$\sigma=\sigma$ for log-Normal$(\mu,\sigma^2)$ baseline.

The results are summarized in Table 1-3 for three different parametric baselines.

\begin{table}[ht]\label{tab:exp}
\caption{Proportional Harzard Model with Exponential baseline}
\centering
\begin{tabular}{rrrrrr}
  \hline
  \hline
  Predictors & $\bbeta$ & HR & lower .95 & upper .95 & p-value \\ 
  \hline
AGE & -0.0344 & 1.0350 & 1.0272 & 1.0429 & $<0.0001$ \\ 
  GENDER(Male) & -0.3504 & 1.4197 & 1.1908 & 1.6926 & 0.0001 \\ 
  NEUTRO & -0.0273 & 1.0276 & 0.9946 & 1.0618 & 0.1025 \\ 
  PLATE & 0.0024 & 0.9976 & 0.9969 & 0.9984 & $<0.0001$ \\ 
  NIH & 0.6682 & 0.5127 & 0.3780 & 0.6953 & $<0.0001$ \\ 
   \hline
\end{tabular}
\end{table}


\begin{table}[ht]\label{tab:wei}
\caption{Proportional Harzard Model with Weibull baseline}
\centering
\begin{tabular}{rrrrrr}
  \hline
  \hline
  Predictors & $\bbeta$ & HR & lower .95 & upper .95 & p-value \\ 
  \hline
AGE & -0.0375 & 1.0327 & 1.0249 & 1.0406 & $<0.0001$ \\ 
  GENDER(Male) & -0.3818 & 1.3876 & 1.1633 & 1.6552 & 0.0003 \\ 
  NEUTRO & -0.0319 & 1.0278 & 0.9957 & 1.0608 & 0.0898 \\ 
  PLATE & 0.0026 & 0.9978 & 0.9971 & 0.9985 & $<0.0001$ \\ 
  NIH & 0.7021 & 0.5475 & 0.4040 & 0.7419 & 0.0001 \\ 
   \hline
  Scale=1.17\\
\hline
\end{tabular}
\end{table}


\begin{table}[ht]\label{tab:log}
\caption{Proportional Harzard Model with log-Normal baseline}
\centering
\begin{tabular}{rrrrrr}
  \hline
  \hline
 Predictors & $\bbeta$ & HR & lower .95 & upper .95 & p-value \\ 
  \hline
AGE & -0.0371 & 1.0234 & 1.0176 & 1.0292 & $<0.0001$ \\ 
  GENDER(Male) & -0.3926 & 1.2773 & 1.1013 & 1.4814 & 0.0012 \\ 
  NEUTRO & -0.0499 & 1.0316 & 1.0089 & 1.0548 & 0.0061 \\ 
  PLATE & 0.0031 & 0.9981 & 0.9976 & 0.9986 & $<0.0001$ \\ 
  NIH & 0.5707 & 0.7006 & 0.5569 & 0.8815 & 0.0024 \\ 
   \hline
  Scale=1.6\\
\hline
\end{tabular}
\end{table}

\section{Discussion}

In Table 1-3, the estimate of $\bbeta$ and corresponding hazard ratio are
similar. The baselines are adjusted for average age, Neutro (ANC) and platelets. In terms of significance at 95\%, Neutro (ANC) is not significant for
Exponential and Weibull cases, although all HRs of Neutro(ANC) are greater than
1. Among all of these models, the 2nd model with Weibull baseline is the closest one
to Cox proportional hazard model with unknown baseline that we fitted in HW \#4.
Therefore, we focus on Weibull model and try to give some clinical interpretation.

Clinically speaking, after adjusting for age, Neutro (ANC), platelets, 
treatment (NIH) and baseline scaling effect (of Weibull and log-Normal), Male
has much higher risk than female. Not surprisingly, age can increase the risk.
Neutro (ANC) is not significant. Concentration of platelets is helpful but
clinically expensive to implement. The IST (immunosupressive treatment) at NIH
can greatly reduce the hazard rate. For Weibull model, it is also an Accelerated
Failure Time (AFT) model. The IST has acceleration factor of $e^{-\hat\beta_5}
= 0.4955$.


\appendix

\section{R Code}

Code is avaliable on \url{https://github.com/mr0112358/SurvivalAnalysis-IMRAW-IST/blob/master/Assignment5.R}

\begin{lstlisting}
# Read and centralize data
library(survival)
IMRAWIST = data.frame(readxl::read_excel("IMRAWandISTnov.xls"))
IMRAWIST$AGE = scale(IMRAWIST$AGE, center = TRUE, scale = FALSE)
IMRAWIST$NEUTRO = scale(IMRAWIST$NEUTRO, center = TRUE, scale = FALSE)
IMRAWIST$PLATE = scale(IMRAWIST$PLATE, center = TRUE, scale = FALSE)

# Function for summary table
sumtable = function(survregobj){
  survregSum = summary(survregobj)
  table = cbind(survregSum$table[2:6,1],
    exp(-survregSum$table[2:6,1]/survregobj$scale), 
    exp(-(survregSum$table[2:6,1] + 1.96*survregSum$table[2:6,2])/survregobj$scale), 
    exp(-(survregSum$table[2:6,1] - 1.96*survregSum$table[2:6,2])/survregobj$scale),
    survregSum$table[2:6,4])
  colnames(table) = c("beta", "hazard ratio", "lower CI", "upper CI", "p-value")
  table
}

# Cox Model for exponential hazard baseline
regfit1 <- survreg(Surv(time = survival, event = DIED, type = "right") 
                ~ AGE + GENDER + NEUTRO + PLATE + NIH,
                data = IMRAWIST,
                dist = "exponential")
sumtable(regfit1)

# Cox Model for Weibull hazard baseline
regfit2 <- survreg(Surv(time = survival, event = DIED, type = "right") 
                ~ AGE + GENDER + NEUTRO + PLATE + NIH,
                data = IMRAWIST,
                dist = "weibull")
sumtable(regfit2)

# Cox Model for log-Normal hazard baseline
regfit3 <- survreg(Surv(time = survival, event = DIED, type = "right") 
                ~ AGE + GENDER + NEUTRO + PLATE + NIH,
                data = IMRAWIST,
                dist = "lognormal")
sumtable(regfit3)
\end{lstlisting}
\end{document}
